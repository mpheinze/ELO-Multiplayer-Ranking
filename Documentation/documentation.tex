\documentclass[12pt]{article}

\usepackage[a4paper, hmargin={2cm, 2cm}, vmargin={2cm, 2cm}]{geometry}
\usepackage{amsmath} % Til generel matematik
\usepackage{setspace} % Til margin specifikation
\usepackage[utf8]{inputenc} % Til ÆæØøÅå
\usepackage{multirow}
\usepackage{caption}
\usepackage[bottom]{footmisc}
\usepackage{array}
\usepackage{tabularx}
\usepackage{csquotes}
\usepackage{enumerate}
\usepackage[hidelinks]{hyperref}
\usepackage{tikz}
\usepackage{amssymb}
\usepackage{pdflscape}
\usepackage{graphicx}
\usepackage{epstopdf}
\usepackage{relsize}

\DeclareGraphicsExtensions{.png,.eps}

\newcolumntype{L}[1]{>{\raggedright\let\newline\\\arraybackslash\hspace{0pt}}m{#1}}
\newcolumntype{C}[1]{>{\centering\let\newline\\\arraybackslash\hspace{0pt}}m{#1}}
\newcolumntype{R}[1]{>{\raggedleft\let\newline\\\arraybackslash\hspace{0pt}}m{#1}}

\setlength{\parindent}{0pt}
\setlength{\extrarowheight}{2pt}
\onehalfspacing


\begin{document}

\title{Multiplayer Elo-Ranking System for BB\&B}
\author{Mathias Pedersen Heinze}
\date{\today}
\maketitle

\begin{abstract}
	This note documents the implementation of the Elo-ranking system for the BB\&B board game club. It is a straight forward implementation of the well-known Elo-ranking system, developed to rank chess players. We apply the standard trick of permuting all participants in multi-player settings.
\end{abstract}
\vspace{20pt}

The new rank $R'$ for player $A$ is calculated as

\begin{equation}
	R'_A=R_A+K(S_A-E_A)
\end{equation}

where $R_A$ is player $A$'s previous rank, $K$ is a weighting factor, $S_A$ is the score of the player, and $E_A$ is the predicted probability of player $A$ winning the game. The score, $S_A$ depends on the number of participants in the game. Calculate the combinations as

\begin{equation}
	\rho=\frac{n(n-1)}{2},
\end{equation}

which is the total number of one-on-one match ups in the game. $S_A$ is then given by

\begin{equation}
	S_A=\frac{1}{\rho}(n-\alpha_A),
\end{equation}

where $\alpha$ is the score of player $A$ in the game (i.e. \textit{first}, \textit{second}, \textit{third}...). Subtract this score by the expected score of the player, given by

\begin{equation}
	E_A=\frac{1}{\rho}\mathlarger{\mathlarger{\sum}}_{1\geq i \geq n,i\ne A}\bigg[1+b^{(R_i-R_A)/c}\bigg]^{-1}.
\end{equation}

Finally, the $K$ factor is a function of the relative size of the game in terms of participants and the number of games played by player $A$

\begin{equation}
	K=\frac{a}{m_A^{1/2}+(N-n)^2},
\end{equation}

where $m$ is the total number of previous games played by player $A$, $N$ is the maximum game size, and $n$ is once again the number of players in the specific game. $a$, $b$ and $c$ are constants to be calibrated.





\end{document}
